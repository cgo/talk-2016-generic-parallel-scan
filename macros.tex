%% To do: trim these definitions

\newcommand\nc\newcommand
\nc\rnc\renewcommand

\usepackage{epsfig}
\usepackage{latexsym}

\nc\out[1]{}

\nc\mynoteOut[2]{\mynote{#1}\out{#2}}

% While working, use these defs
%% \nc\mynote[1]{{\em [#1]}}
%% \nc\mynotefoot[1]{\footnote{\mynote{#1}}}
% But for the submission, use these
\nc\mynote\out
\nc\mynotefoot\out

\nc\todo{\mynote{To do.}}

\nc\figlabel[1]{\label{fig:#1}}
\nc\figref[1]{Figure~\ref{fig:#1}}

\nc\needcite{\mynote{ref}}

% \nc{\Opid}[1]{\operatorname{#1}}
\nc{\Opid}[1]{\Varid{#1}}
\nc{\Varap}[1]{\Opid{#1}\,}
\nc{\Varapp}[2]{\Varap{#1}{(#2)}}

\nc\wpicture[2]{\includegraphics[width=#1]{#2}}

\nc\wfig[2]{
\begin{center}
\wpicture{#1}{#2}
\end{center}
}
\nc\fig[1]{\wfig{4in}{#1}}

\nc\usebg[1]{\usebackgroundtemplate{\wpicture{1.2\textwidth}{#1}}}

\nc\framet[2]{\frame{\frametitle{#1}#2}}

\nc\hidden[1]{}


\newcommand{\Pair}{\Varid{Pair}}

\newcommand{\stat}[6]{
#1 & #2 & #3 & #4 & #5 & #6 \\ \hline
}
\newcommand{\fftStats}[1]{
\begin{center}
\begin{tabular}{|c|c|c|c|c|c|}
  \hline
  \stat{Type}{$+$}{$\times$}{$-$}{total}{max depth} \hline
  #1
  \hline
\end{tabular}
\end{center}
}

\nc{\subo}{_{\!\mathit{1}}}

\nc\partframe[1]{\framet{}{\begin{center} \vspace{6ex} {\Huge \textcolor{partColor}{#1}} \end{center}}}

\nc\symTwo[1]{\mathbin{#1\!\!\!#1}}
\nc\symThree[1]{\mathbin{#1\!\!#1\!\!#1}}


\usefonttheme{serif}
\usepackage{framed}
\usepackage{hyperref}
\usepackage{color}

\definecolor{linkColor}{rgb}{0,0.42,0.3}
\definecolor{partColor}{rgb}{0,0,0.8}

\hypersetup{colorlinks=true,urlcolor=linkColor}

\usepackage{graphicx}
\usepackage{color}
\DeclareGraphicsExtensions{.pdf,.png,.jpg}

\usepackage{geometry}
% \usepackage[a4paper]{geometry}

%% \usepackage{wasysym}
\usepackage{mathabx}
\usepackage{setspace}
\usepackage{enumerate}
\usepackage{tikzsymbols}
% \usepackage{fancybox}
\usepackage[many]{tcolorbox}

\tcbset{enhanced,boxrule=0.5pt,colframe=black!50!blue,colback=white,boxsep=-2pt,drop fuzzy shadow}

\usepackage[absolute,overlay]{textpos}  % ,showboxes

\TPGrid{364}{273} %% roughly page size in points

\useinnertheme[shadow]{rounded}
% \useoutertheme{default}
\useoutertheme{shadow}
\useoutertheme{infolines}
% Suppress navigation arrows
\setbeamertemplate{navigation symbols}{}

\newcommand\sourced[1]{\href{#1}{\tiny (source)}}


\definecolor{statColor}{rgb}{0,0.5,0}

\newcommand{\stats}[2]{
{\small \textcolor{statColor}{work: #1, depth: #2}}}

\newcommand\ccircuit[3]{
\framet{#1}{
\vspace{#2ex}
\wfig{4.5in}{circuits/#3}
}}

\newcommand\circuit[5]{
\ccircuit{#1 \hfill \stats {#4}{#5}\hspace{2ex}}{#2}{#3}
}

\DeclareMathOperator{\D}{Depth}
\DeclareMathOperator{\W}{Work}
\nc\Size[1]{\lvert #1 \rvert}


\author{\href{http://conal.net}{Conal Elliott}}
\institute{Target}


\graphicspath{{Figures/}}

\definecolor{shadecolor}{rgb}{0.95,0.95,0.95}
\setlength{\fboxsep}{0.75ex}
\setlength{\fboxrule}{0.15pt}
%% \setlength{\shadowsize}{2pt}

%% \nc\cbox[1]{\raisebox{-0.5\height}{\fbox{#1}}}
\nc\cpic[2]{\fbox{\wpicture{#1}{circuits/#2}}}
\nc\ccap[3]{
\begin{minipage}[c]{0.48\textwidth}
\begin{center}
\cpic{#2}{#3}\par\vspace{0.5ex}#1\par
\end{center}
\end{minipage}
}

\setlength{\itemsep}{2ex}
\setlength{\parskip}{1ex}
% \setstretch{1.2} % ??
