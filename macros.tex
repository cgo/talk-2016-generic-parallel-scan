%% To do: trim these definitions

\newcommand\nc\newcommand
\nc\rnc\renewcommand

\usepackage{epsfig}
\usepackage{latexsym}

\nc\out[1]{}

\nc\mynoteOut[2]{\mynote{#1}\out{#2}}

% While working, use these defs
%% \nc\mynote[1]{{\em [#1]}}
%% \nc\mynotefoot[1]{\footnote{\mynote{#1}}}
% But for the submission, use these
\nc\mynote\out
\nc\mynotefoot\out

\nc\todo{\mynote{To do.}}

\nc\figlabel[1]{\label{fig:#1}}
\nc\figref[1]{Figure~\ref{fig:#1}}

\nc\needcite{\mynote{ref}}

% \nc{\Opid}[1]{\operatorname{#1}}
\nc{\Opid}[1]{\Varid{#1}}
\nc{\Varap}[1]{\Opid{#1}\,}
\nc{\Varapp}[2]{\Varap{#1}{(#2)}}

\nc\wpicture[2]{\includegraphics[width=#1]{#2}}

\nc\wfig[2]{
\begin{center}
\wpicture{#1}{#2}
\end{center}
}
\nc\fig[1]{\wfig{4in}{#1}}

\nc\usebg[1]{\usebackgroundtemplate{\wpicture{1.2\textwidth}{#1}}}

\nc\framet[2]{\frame{\frametitle{#1}#2}}

\nc\hidden[1]{}


\newcommand{\Pair}{\Varid{Pair}}

\newcommand{\stat}[6]{
#1 & #2 & #3 & #4 & #5 & #6 \\ \hline
}
\newcommand{\fftStats}[1]{
\begin{center}
\begin{tabular}{|c|c|c|c|c|c|}
  \hline
  \stat{Type}{$+$}{$\times$}{$-$}{total}{max depth} \hline
  #1
  \hline
\end{tabular}
\end{center}
}

\nc{\subo}{_{\!\mathit{1}}}

\nc\partframe[1]{\framet{}{\begin{center} \vspace{6ex} {\Huge \textcolor{partColor}{#1}} \end{center}}}
